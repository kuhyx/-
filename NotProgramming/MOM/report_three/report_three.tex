\documentclass[12pt]{article}
\usepackage[T1]{fontenc}
\usepackage[polish]{babel}
\usepackage[utf8]{inputenc}
\usepackage{hyperref}
\usepackage{listings}
\title{Modelowanie Matematyczne, Projekt 3, Dane nr 1.6}
\author{Krzysztof Rudnicki, 307585}
\setcounter{section}{-1}
\begin{document}
\maketitle 
\section{Wstęp} 
2 Fabryki, F1, F2 \\ 
4 Magazyny, M1, M2, M3, M4 \\ 
6 Klientów, K1, K2, K3, K4, K5, K6 \\ 
\paragraph{Koszty dystrybucji towaru}

\begin{center}
    \begin{tabular}{ | c | c | c | c | c | c | c |  }
        \hline
     zaopatruje & F1 & F2 & M1 & M2 & M3 & M4  \\ 
     \hline
     Magazyny &  &  &  &  &  &   \\ 
     \hline
     M1 & 0.3 & - &  &  &  &   \\ 
     \hline
     M2 & 0.5 & 0.4 &  &  &  &   \\ 
     \hline
     M3 & 1.2 & 0.5 &  &  &  &   \\ 
     \hline
     M4 & 0.8 & 0.3 &  &  &  &   \\ 
     \hline
     Klientów &  &  & &  &  & \\ 
     \hline
     K1 & 1.2 & 1.8 & - & 1.4 & - & -    \\ 
     \hline
     K2 & - & - & 1.2 & 0.3 & 1.3 & -  \\ 
     \hline
     K3 & 1.2 & - & 0.2 & 1.8 & 2.0 & 0.4  \\ 
     \hline
     K4 & 2.0 & - & 1.7 & 1.3 & - & 2.0  \\ 
     \hline
     K5 & - & - & - & 0.5 & 0.3 & 0.5  \\ 
     \hline
     K6 & 1.1 & - & 2.0 & - & 1.4 & 1.5  \\ 
     \hline
    \end{tabular}
    \end{center}
\section{Model Dwukryterialny}
\paragraph{Zbiory}
\begin{itemize}
    \item $i, k \in F$ - Fabryki 
    \item $i, l \in M$ - Magazyny 
    \item $j \in K$ - Klienci
\end{itemize}
\paragraph{Parametery}
\begin{itemize}
    \item $C_{i, j}$ - Koszty transportu dóbr z punktów $i$ (fabryka lub magazyn) do klienta $j$
    \item $C_{k, l}$ - Koszty transportu dóbr z fabryki $i$ do magazynu $k$ 
    \item $P_{i, j}$ - Binarnie określa czy preferencja klienta zostąła spełniona (1) czy nie (0)
    \item $S_j$ - Poziom satysfakcji klienta $j$
    \item $D_j$ - Zapotrzebowanie klienta $j$
\end{itemize}
\paragraph{Zmienne decyzyjne}
\begin{itemize}
    \item $x_{i,j}$ - Liczba dóbr (w tys. ton) przetransportowana z punktu $i$ (fabryka lub magazyn) do klienta $j$
    \item $y_{k, l}$ - Liczba dóbr (w tys. ton) przetransportowana z fabryki $k$ do magazynu $l$
\end{itemize} 
\paragraph{Funkcja celu}
\begin{enumerate}
    \item Minimalizacja kosztów dystrybucji \\ 
    \[ Min(\sum_{i, j} C_{i, j} * x_{i, j} + \sum{i, k} C_{i, k} * y_{i, k}) \]
    \item Maksymalizacja satysfakcja klienta \\ 
    W celu maksymalizacji satysfakcji klienta policzymy ile z dostaw do klientów odbyło się z preferencyjnych źródeł \\
    \[ Max(\sum_{j} S_j * P_{i,j} * x_{i, j}) \] 
\end{enumerate}
\paragraph{Ograniczenia}
Miesięczne możliwości produkcyjne fabryk
\begin{equation}
    \sum_j x_{F1, j} + \sum_k y_{F1, k} \leq 150
\end{equation}
\begin{equation}
    \sum_j x_{F2, j} + \sum_k y_{F2, k} \leq 200
\end{equation}
Miesięczna ilośc obsługiwanego towaru przez magazyny 
\begin{equation}
    \sum_j x_{M1, j} \leq 70
\end{equation}
\begin{equation}
    \sum_j x_{M2, j} \leq 50
\end{equation}
\begin{equation}
    \sum_j x_{M3, j} \leq 100
\end{equation}
\begin{equation}
    \sum_j x_{M4, j} \leq 40
\end{equation}
Spełnienie preferencji klienta 
\begin{equation}
    \sum_i x_{i, j} = D_j
\end{equation}
Wartości niezerowe 
\begin{equation}
    x_{i, j}, y_{i, k} \geq 0
\end{equation}

\section{Implementacja}
Do implementacji użyty został python z biblioteką pulp \href{https://coin-or.github.io/pulp/index.html}{https://coin-or.github.io/pulp/index.html} \\ 
Dzięki temu wykorzystujemy zarówno łatwość pythona jak i możliwości używania różnych solverów (CBC, GLPK, CPLEX, Gurobi...) przez pulpa \\ 
\begin{lstlisting}[language=Python, caption={Import bilbioteki pulp}]
    import pulp
\end{lstlisting}
\begin{lstlisting}[language=Python, caption={Iniicjalizacja modelu}]
    model = pulp.LpProblem("Optimal_Distribution", pulp.LpMinimize)
\end{lstlisting}

\begin{lstlisting}[language=Python, caption={Zmienne decyzyjne}]
    x =
    pulp.LpVariable.dicts("x",
    [(i, j) for i in punkty for j in klienci],
    lowBound=0,
    cat='Integer')
    y 
     pulp.LpVariable.dicts("y",
     [(i, k) for i in fabryki for k in magazyny],
     lowBound=0,
     cat='Integer')
\end{lstlisting}

\begin{lstlisting}[language=Python, caption={Funkcje celu}]
    # Objective function components
    koszt_dystrybucji =
    pulp.lpSum(
        [cost[i][j] * x[(i, j)] for i in punkty for j in klienci]
        )
    koszt_magazynowania =
    pulp.lpSum(
        [cost[i][k] * y[(i, k)] for i in fabryki for k in magazyny]
        )

    # Define objective
    model += 
    alpha
    * (koszt_dystrybucji + koszt_magazynowania)
    - beta * poziom_satysfakcji    
\end{lstlisting}

\begin{lstlisting}[language=Python, caption={Ograniczenia}]
for i in fabryki:
    model +=
        pulp.lpSum([x[(i, j)] for j in klienci]
        + [y[(i, k)] for k in magazyny])
        <= mozliwosci_fabryki[i]
for k in magazyny:
    model +=
    pulp.lpSum([x[(k, j)] for j in klienci])
    <= mozliwosci_magazynu[k]
for j in klienci:
    model +=
    pulp.lpSum([x[(i, j)] for i in punkty]) == wymagania_klienta[j]
\end{lstlisting}

\begin{lstlisting}[language=Python, caption={Rozwiązanie problemu}]
    model.solve()
\end{lstlisting}
\

\begin{lstlisting}[language=Python, caption={Przedstawienie wyników}]
for v in model.variables():
    print(v.name, "=", v.varValue)
\end{lstlisting}

\section{Rozwiązanie efektywne}
Aby zdefiniować rozwiązanie efektywne sprawdzamy sumaryczy obu funkcji celu dla wartości $\alpha$ i $beta$ od 1 do 10 \\ 
w tym celu napisany został kod który modyfikuje wartości $\alpha$ i $\beta$ w pętli 

\begin{lstlisting}[language=Python, caption={Wyznaczanie alpha i beta}]

# Varying alpha and beta
for alpha in range(0, 11):
    beta = 10 - alpha
    alpha /= 10.0
    beta /= 10.0

    # Update objective function
    model.objective = alpha * cost_distribution - beta * satisfaction_component

    # Solve the model
    model.solve()

    # Record the results
    total_cost = value(cost_distribution)
    total_satisfaction = value(satisfaction_component)
    cost_results.append(total_cost)
    satisfaction_results.append(total_satisfaction)
\end{lstlisting}

    \end{document}